\chapter{Dataset}

\section{Dataset Acquisition}
\label{section:dataset-exploration}

\textbf{FIXME}{This is not the transition model, it's the dataset used for transitions (maybe transition dataset?}
\subsection{Sentences Dataset}
\label{subsection:sencences_ds}
We performed our experiments on two different transition models.\\
The first one is a concatenation of public domain book excerpts from 
\href{http://www.gutenberg.org/wiki/Main\_Page}{ 
	\textcolor{blue}{Project Gutenberg}}, containing about a million words. 

\textbf{FIXME}:{
The second one has been extracted from the collection  
\href{https://www.kaggle.com/mokosan/lord-of-the-rings-character-data#LordOfTheRingsBook.json}{ 
	\textcolor{blue}{LordOfTheRingsBook.json}} containing the collections of the “Lord Of The Rings” books.}

To each corpora we applied some preprocessing procedures, in particular we divided them in lower-case 
sentences, and removed special characters and punctuation, obtaining the datasets \texttt{big\_clean.csv} and 
\texttt{lotr\_clean.csv} (\textbf{FIXME }something about apostrophes)


\subsection{Typos Dataset}
The basic typos dataset was collected from the following resources:
%\footnote{\url{https://www.dcs.bbk.ac.uk/~ROGER/corpora.html}}  
%\footnote{\url{https://www.kaggle.com/rtatman/spelling-variation-on-urban-dictionary}}  
%\footnote{\url{https://www.kaggle.com/bittlingmayer/spelling}}
%\footnote{\url{http://luululu.com/tweet}}
\begin{itemize}
	\item \textsc{birkbeck} \footlabel{note1}{\url{https://www.dcs.bbk.ac.uk/~ROGER/corpora.html}} : contains 
	\num{36133} misspellings of \num{6136} words, taken from the native-speaker section (British and 
	American) of the Birkbeck spelling error corpus.
	\item \textsc{holbrook} \footref{note1}: contains \num{1791} misspellings of \num{1200} words, taken from 
	the book "English for the Rejected" by David Holbrook (Cambridge University Press - 1964).
	\item \textsc{aspell} \footref{note1} :contains \num{531} misspellings of \num{450} words, taken from one 
	assembled by Atkinson for testing the GNU Aspell spellchecker.
	\item \textsc{wikipedia} \footref{note1}: contains \num{2455} misspellings of \num{1922} words, taken from 
	the misspellings made by Wikipedia editors.
	\item \textsc{urban-dictionary-variants} 
	\footnote{\url{https://www.kaggle.com/rtatman/spelling-variation-on-urban-dictionary}}  : contains 
	\num{716} variant 
	spellings, taken from the text scraped from Urban Dictionary (in UK English).
	\item \textsc{spell-set} \footnote{\url{https://www.kaggle.com/bittlingmayer/spelling}}: contains 
	\num{670} typos.
	\item \textsc{tweet-typo} \footnote{\url{http://luululu.com/tweet}}:: contains \num{39172} typos, taken 
	from Twitter.
\end{itemize}

All the datasets are cleaned and joined in a new one that contains \num{79677} rows, each with a typo and the 
corresponding correct word.
The dataset is then divided into two corpora: \num{80}\% is used as a train set (\num{63679} rows) and 
\num{20}\% is used as a test set (\num{15998} rows).


\textcolor{red}{
	We decided to created another dataset of typos starting from the \\\texttt{lotr\_clean.csv}. 
	Extracting a list of all the words contained in this corpus, for each of these we have generated a sequence of 
	five typos according to the algorithm that will be defined in the chapter \ref{subsection:perturbed}.
	The final dataset contains \num{62759} row, with the same structure as the one described above.
	Also in this case the two datasets train and test were created, respectively containing \num{50058} and 
	\num{12701}\% rows.}

\subsection{Language Dataset}
A language model represents the frequency of words in a certain language.
We used two different language model datasets. \\
The first one is a lists of most frequent words from 
\href{https://en.wiktionary.org/wiki/Wiktionary:Frequency_lists}{ 
	\textcolor{blue}{Wiktionary}} and the 
	\href{http://www.kilgarriff.co.uk/bnc-readme.html}{\textcolor{blue}{British National Corpus}}. 
We use \texttt{frequency-alpha-gcide.txt}, a smaller version derived from the original dataset 
\href{https://books.google.com/ngrams/}{\textcolor{blue}{Google's ngram corpora}}, that includes wordlists, 
cleaned up and limited to only the top \num{65537} words.

We found some problems with this dataset, for example the lack of proper names, city names, countries, brands 
etc.
Moreover, most of the typical words of the language used in the sentence dataset were missing.
For this reason, we decided to create a new language model \texttt{lotr\_language\_model.txt}, based on the 
frequency of the word in the dataset \texttt{lotr\_clean.csv} \textbf{FIXME:}(12506 parole).

\textbf{FIXME}{
	Each of these datasets contains in each row the word itself and a percentage of how often each word was 
	being used.}

\subsection{Perturbated Dataset}
\label{subsection:perturbed}
In order to evaluate our algorithm on whole sentences, we create new perturbed datasets starting from the 
datasets \texttt{big\_clean} and \texttt{lotr\_clean} described in the section \ref{subsection:sencences_ds}.

\textbf{FIXME}:Estimates for the frequency of spelling errors in human-typed text vary from 1-2\% for carefully 
retyping already printed text to 10-15\% for web queries.
The disturbance introduced presents an error dependent on the error model previously presented, with the 
difference that it is created starting from the typos belonging to the test datasets.
\textbf{FIXME}:{Analysis of spelling error data has shown that the majority of spelling errors consist of a 
single-letter change and so we often make the simplifying assumption that these candidates have an edit 
distance of 1 from the error model.}

Three different texts for each dataset, of approximately \num{50000} sentences each, have been generated, 
each of which has a percentage of errors in the text of \textbf{FIXME} \num{10}-\num{15}-\num{20}\% 
respectively. \\

We implemented a perturbation algorithm, which for each line of our input file generates a new perturbed 
string.\\
The input text is perturbed accordingly the following steps:

\begin{enumerate}
	\item The probability that a word has an edit is computed by multiplying the value of $p$, coming from the 
	error model, by the percentage of errors desired (10-15-20\%).
	\item For each word of length $n$, the number of edits to be introduced $x$ is calculated according to the 
	relation $x \sim \text{Bin}(n, p)$
	\item The characters to be changed within each word are chosen randomly.
	\item \textbf{FIXME}: The type of edit to be introduced within each word is chosen according to the 
	probability of each type of error. 
	The disturbance goes to alter the letters designated not in a random manner. 
	In fact, we use four different probabilities to define whether a letter will be deleted from the index in question, 
	if a new letter will be inserted after the actual character, or if the current character will be replaced with one of 
	the possible letters according to the error model probability, or if the current character will be swapped with 
	the next or the previous one.
	
\end{enumerate}

Swap errors are only introduced if there are no further changes in the word. Cases of elimination of a whole 
word are excluded, as these would heavily influence the evaluation metrics as they are inconsistent with our 
model. 

\subsection{On using consistent languages}
Our experimentation with different datasets has made us come to the conclusion that, for the models to work 
optimally, each dataset must share the same language. That is, the datasets must share most of the vocabulary 
used in them. We have noticed that when the datasets don't share the same vocabulary, and the distribution of 
words isn't aligned between them, the error model might produce candidates that it found most probable for 
the language model he's trained for, but in the HMM, if trained on a transitions dataset not sharing the same 
language as the language model, might discard these candidates for less-likely ones that have a higher 
transition probability.


%(\num{58000} sentence)

%\begin{lstlisting}[
%label={code:perturbation-algorithm},
%caption={Text perturbation algorithm},
%captionpos=b,
%breaklines=true,                                    
%language=Python,
%frame=ltrb,
%framesep=5pt,
%basicstyle=\small,
%keywordstyle=\ttfamily\color{OliveGreen},
%identifierstyle=\ttfamily\color{MidnightBlue}\bfseries,
%commentstyle=\color{Brown},
%stringstyle=\ttfamily,
%showstringspaces=false
%]
%def perturb():
%	# Create a model for the test set
%	hmm = HMM(1, max_edits=2, max_states=3)
%	hmm.train(words_ds="../data/word_freq/frequency-alpha-gcide.txt",
%					   sentences_ds="../data/texts/big_clean.txt",
%					   typo_ds="../data/typo/clean/test.csv")
%	
%	cleaned = open("../data/texts/big_clean.txt", "r")
%	
%	if not os.path.exists("../data/texts/perturbated/"):
%		os.makedirs("../data/texts/perturbated/")
%	
%	perturbed = open("../data/texts/perturbated/big_perturbed.txt", "w")
%	
%	# probability that a word has an edit
%	p = hmm.error_model["p"]
%	
%	# probability of the various edit
%	prob_swap = hmm.error_model["swap"]
%	prob_ins = hmm.error_model["ins"]
%	prob_del = hmm.error_model["del"]
%	prob_sub = 1 - (prob_swap + prob_ins + prob_del)
%	
%	edit_prob = [prob_swap, prob_ins, prob_del, prob_sub]
%	
%	for i, e in enumerate(edit_prob):
%		if i == 0:
%			continue
%	
%	edit_prob[i] = edit_prob[i] + edit_prob[i - 1]
%	
%	def substitute(word):
%		l = list(word)
%		if not l[indices[j]] in hmm.error_model["sub"]:
%			l[indices[j]] = random.choice(string.ascii_letters).lower()
%		else:
%			l[indices[j]] = np.random.choice(list(hmm.error_model["sub"][l[indices[j]]].keys()))
%		return "".join(l)
%	
%	for line in cleaned:
%		line_words = line.split()
%	
%		for i, word in enumerate(line_words):
%			n = len(word)
%			# number of errors to introduce in the word
%			x = np.random.binomial(n, p)        # x ~ Bin(p, n)
%
%			# choose two letter to change
%			indices = np.random.choice(n, x, replace=False)
%			indices = -np.sort(-indices)
%			
%			for j in range(x):
%				r = np.random.random()
%				
%				for k, e in enumerate(edit_prob):
%					if r <= edit_prob[k]:
%						break
%					value = k
%
%				# swap if you have to do only one edit
%				if value == 0 and x == 1:
%					# if the letter to switch is the last one, switch with the previous one
%					if len(indices) <= j + 1:
%						word = word[0:indices[j] - 1] + word[indices[j]] + word[indices[j] - 1] +  word[indices[j] + 1:]
%					else:
%						word = word[0:indices[j]] + word[indices[j] + 1] + word[indices[j]] + word[indices[j] + 2:]
%
%				# insert a letter in a random position (after idx)
%				elif value == 1:
%					new_letter = random.choice(string.ascii_letters)
%					word = word[0:indices[j]] + new_letter + word[indices[j] + 1:]
%				
%				# delete a letter
%				elif value == 2:
%					if len(word) == 1:
%						# if the word is 1 char, don't delete the word but substitute it with another one
%						word = substitute(word)
%					else:
%						word = word[0:indices[j]] + word[indices[j] + 1:]
%
%				# substitute a letter
%				else:
%					word = substitute(word)
%
%			line_words[i] = word
%
%		line = " ".join(line_words)
%		perturbed.write(line + '\n')
%				
%	perturbed.close()
%	cleaned.close()
%\end{lstlisting}


