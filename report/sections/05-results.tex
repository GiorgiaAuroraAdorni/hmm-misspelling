\textbf{FIXME} comparison
\textbf{FIXME} performance

The fact that words are generally more frequent than their misspellings can be used in candidate suggestion, by 
building a set of words and spelling variations that have similar contexts, sorting by frequency, treating the most 
frequent variant as the source, and learning an error model from the difference


\textbf{FIXME} Mostrare il nostro modello con p e la tabella delle probabilità
We detected some problems with our dataset, in particular it lacks of plural forms and other things

In order to avoid this problem, we decided to try a new approach that use lemmaisation (stemmisationj ......



Analysis of spelling error data has shown that the majority of spelling errors consist of a single-letter change and 
so we 
often make the simplifying assumption that these candidates have an edit distance of 1 from the error word.



Evaluating spell correction algorithms is generally done by holding out a train- ing, development and test set from 
lists of errors like those on the Norvig and Mitton sites mentioned above.

% how many observation
\begin{figure}[H]
	\centering
	\begin{tabular}{lcccc}
		\toprule
		 & Time (sec)  & Accuracy Top1 & Accuracy Top3  &  Accuracy Top5 \\
		\midrule
		Train & \num{20985,70} & 41,38\%  & 57,28 \% & 61,60 \%  \\
		Test &	\num{5270,06}  & 41,54\%  & 57,18 \% & 61,15 \%  \\
		\bottomrule
	\end{tabular}
	\captionof{table}{Typos performance evaluation}
	\label{tab:typo-eval}
\end{figure}


\begin{figure}[H]
	\centering
	\begin{tabular}{lcccc}
		\toprule
		& Time (sec)  & Accuracy & Initial Error  &  Accuracy Top5 \\
		\midrule
		Test &	\num{2695,01}  & 47,99\%  & 15,01\% & 61,15 \%  \\
		\bottomrule
	\end{tabular}
	\captionof{table}{Sentences performance evaluation}
	\label{tab:sentence-eval}
\end{figure}