\textbf{FIXME} comparison
\textbf{FIXME} performance

We performed three different types of experiments. \\

The first one using as a transition model the dataset \texttt{big\_clean}, the associate perturbed dataset and test error 
models, and the language model \texttt{frequency-alpha-gcide}.

A second one using the same datasets but introducing a lemmatisation consisting in a simple dictionary lookup.

The last one using as transition model the dataset \texttt{lotr\_clean}, the associate perturbed dataset and test error 
models, and the language model \texttt{lotr\_language\_model}.\\

In all the experiments to come, reference will be made to the following variables:
\begin{itemize}
	\item p: the probability that a word has an edit
	\item ins: the probability that a word has a letter insertion
	\item del: the probability that a word has a letter deletion
	\item sub: the probability that a word has a letter substitution
	\item swap: the probability that a word has a swap between two letters
\end{itemize}

\subsection{Experiment 1}

\textbf{FIXME}
Edit distance, perturbation dataset used


\begin{figure}[H]
	\centering
	\begin{tabular}{ccccc}
		\toprule
		p 				 & ins 				 	& del  				&  sub 			   &   swap\\ \midrule
		\num{0.5} & \num{0,70} & \num{0,70}  & \num{0,70} & \num{0,70}\\
		\bottomrule
	\end{tabular}
	\captionof{table}{Error Model}
	\label{tab:error_model1}
\end{figure}

% how many observation
\begin{figure}[H]
	\centering
	\begin{tabular}{lcccc}
		\toprule
		& Time (sec)  & Accuracy Top1 & Accuracy Top3  &  Accuracy Top5 \\
		\midrule
		Train & \num{} & \%  &  \% &  \%  \\
		Test &	\num{1220}  & \num{40,96}\%  & \num{56,34} \% & \num{60,03} \%  \\
		\bottomrule
	\end{tabular}
	\captionof{table}{Typos performance evaluation}
	\label{tab:typo-eval1}
\end{figure}


\begin{figure}[H]
	\centering
	\begin{tabular}{ccccccc}
		\toprule
		\#sentence & Time (sec)  & Accuracy & Initial Error  &  Precision & Recall & Specificity \\
		\midrule
		\num{5000}	& \num{3549}  & \num{44,40}\%  & \num{14,77}\% & \num{89,43}\% & \num{45,42}\%  & 
		\num{14,81}\%  
		\\
		\bottomrule
	\end{tabular}
	\captionof{table}{Sentences performance evaluation}
	\label{tab:sentence-eval1}
\end{figure}

\subsection{Experiment 2}
\begin{figure}[H]
	\centering
	\begin{tabular}{ccccc}
		\toprule
		p 				 & ins 				 	& del  				&  sub 			   &   swap\\ \midrule
		\num{0.5} & \num{0,70} & \num{0,70}  & \num{0,70} & \num{0,70}\\
		\bottomrule
	\end{tabular}
	\captionof{table}{Error Model}
	\label{tab:error_model2}
\end{figure}

We detected some problems with our dataset, in particular it lacks of plural forms and other things

In order to avoid this problem, we decided to try a new approach that use lemmaisation (stemmisationj ......

\begin{figure}[H]
	\centering
	\begin{tabular}{lcccc}
		\toprule
		& Time (sec)  & Accuracy Top1 & Accuracy Top3  &  Accuracy Top5 \\
		\midrule
		Train & \num{20986} & \num{41,38}\%  & \num{57,28} \% & \num{61,60} \% \\
		Test &	\num{5270}  & \num{41,54}\%  & \num{57,18} \% & \num{61,15} \%  \\
		\bottomrule
	\end{tabular}
	\captionof{table}{Typos performance evaluation}
	\label{tab:typo-eval2}
\end{figure}


\begin{figure}[H]
	\centering
	\begin{tabular}{ccccccc}
		\toprule
		\#sentence & Time (sec)  & Accuracy & Initial Error  &  Precision & Recall & Specificity \\
		\midrule
		\num{1000}	& \num{2705}  & \num{53,46}\%  & \num{15,01}\% & \num{90,96}\% & \num{54,50}\%  & 
		\num{10,52}\%  
		\\
		\bottomrule
	\end{tabular}
	\captionof{table}{Sentences performance evaluation}
	\label{tab:sentence-eval2}
\end{figure}

\subsection{Experiment 3}

Edit distance, perturbation dataset used

\begin{figure}[H]
	\centering
	\begin{tabular}{ccccc}
		\toprule
		p 				 & ins 				 	& del  				&  sub 			   &   swap\\ \midrule
		\num{0.5} & \num{0,70} & \num{0,70}  & \num{0,70} & \num{0,70}\\
		\bottomrule
	\end{tabular}
	\captionof{table}{Error Model}
	\label{tab:error_model3}
\end{figure}

Analysis of spelling error data has shown that the majority of spelling errors consist of a single-letter change and 
so we often make the simplifying assumption that these candidates have an edit distance of 1 from the error word.

\textbf{FIXME}
\begin{figure}[H]
	\centering
	\begin{tabular}{lcccc}
		\toprule
		& Time (sec)  & Accuracy Top1 & Accuracy Top3  &  Accuracy Top5 \\
		\midrule
		Train & \num{20986} & \num{41,38}\%  & \num{57,28} \% & \num{61,60} \% \\
		Test &	\num{5270}  & \num{56,78}\%  & \num{74,75} \% & \num{80,59} \%  \\
		\bottomrule
	\end{tabular}
	\captionof{table}{Typos performance evaluation}
	\label{tab:typo-eval3}
\end{figure}


\begin{figure}[H]
	\centering
	\begin{tabular}{ccccccc}
		\toprule
		\#sentence & Time (sec)  & Accuracy & Initial Error  &  Precision & Recall & Specificity \\
		\midrule
		\num{1620}	& \num{2705}  & \num{58,21}\%  & \num{17,39}\% & \num{90,30}\% & \num{58,62}\%  & 
		\num{8,43}\%  
		\\
		\bottomrule
	\end{tabular}
	\captionof{table}{Sentences performance evaluation}
	\label{tab:sentence-eval3}
\end{figure}