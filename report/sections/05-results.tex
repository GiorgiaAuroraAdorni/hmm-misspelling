\chapter{Experiment and Results}

\textbf{FIXME} comparison
\textbf{FIXME} performance

\subsection{Evaluation Metrics}
We evaluate the local model performances, or that carried out on individual typos, through various measures of 
accuracy. 
We compute the \textsc{Top-1 Accuracy} comparing the misspelt word and the best candidate predicted by the 
model. 
Than we also compute the \textsc{Top-3 Accuracy} and \textsc{Top-5 Accuracy}, comparing the misspelt word 
with the first $N$ candidates, in this case \num{3} and \num{5}, given from the model.\\


As regards the performance evaluation of the entire sequences, after every training and prediction we save a csv 
file containing the sentences with the perturbations, the sentences that we want to predict (hidden truth) and the 
sentences provided by the model.

The idea to evaluate the performance of the model is very simple: we scroll through the lines containing the 
perturbed, the predicted and the original (intended) text and, for each word we verify if it was disturbed or not and 
if it corresponds to the original truth.
Therefore, for each word, the following cases can occur:
\begin{enumerate}
	\item \textit{Perturbed word not correctly provided}
	\begin{enumerate}
		\item the word was not the subject of attempted correction by the model
		\item the word has been the subject of attempted correction by the model but without success
	\end{enumerate}
	\item \textit{Perturbed word correctly provided}
	\item \textit{Unperturbed word not correctly provided}
	\item \textit{Unperturbed word correctly provided}
\end{enumerate}

We can therefore represent the confusion matrix of the data in output through table \ref{tab:confmat}:

\begin{figure}[H]
	\centering
	\begin{tabular}{lc|cc}
		\toprule
		& & \multicolumn{2}{c}{\textbf{Model Prediction}}\\
		& & \textsc{True}  & \textsc{False} \\
		\midrule
		\multirow{4}{*}{\textbf{Hidden Truth}} 
		& \multirow{2}{*}{\textsc{True}}   & True Positive & False Negative	\\ 
			& & Case 2. & Case 1(a).	\\ 
		& \multirow{2}{*}{\textsc{False}}  & False Positive & True Negative	\\
		& &  Case 3. \& 1(b).  & Case 4.	\\ 
		\bottomrule
	\end{tabular}
	\captionof{table}{Confusion matrix}
	\label{tab:confmat}
\end{figure}

From the confusion matrix defined in the table above, it is possible to calculate the following performance 
metrics in a simple way:
\begin{itemize}
	\item \textsc{Rate of perturbed words correctly predicted}:
	\[ \frac{\mbox{True Positive}}{\sum \mbox{Perturbed}} = \frac{\mbox{Case 2.}}{\mbox{Case 2.} + \mbox{Case 
	1.}}\]
	\item \textsc{Rate of unperturbed words not correctly predicted}:
	\[ \frac{\mbox{True Negative}}{\sum \mbox{Unperturbed}} = \frac{\mbox{Case 2.}}{\mbox{Case 3.} + 
	\mbox{Case 4.}}\]
	\item \textsc{Accuracy}: measures the goodness of the model among the positive correction results obtained
	\[ \frac{\mbox{True Positive} + \mbox{True Negative}}{\sum \mbox{All}} = \frac{\mbox{Case 2.} + \mbox{Case 
	4.}}{\mbox{Case 1.} + \mbox{Case 2.} + \mbox{Case 3.} + \mbox{Case 4.}}\]
	\item \textsc{Precision}: ratio of the corrected predictions with respect to the significant values returned by the 
	model (the corrections made) 
		\[ \frac{\mbox{True Positive}}{\mbox{True Positive} + \mbox{False Positive}} = \frac{\mbox{Case 
		2.}}{\mbox{Case 1(b).} + \mbox{Case 2.} + \mbox{Case 3.}}\]
	\item \textsc{Recall}: proportion of all the correct results
		\[ \frac{\mbox{True Positive}}{\mbox{True Positive} + \mbox{False Negative}} = \frac{\mbox{Case 
		2.}}{\mbox{Case 1(a).} + \mbox{Case 2.}}\]
	\item \textsc{F-Measure}: weighted average (between 0 and 1) with respect to precision and recall
			\[ 2 * \frac{\mbox{Precision} * \mbox{Recall}}{\mbox{Precision} + \mbox{Recall}} \]
\end{itemize}

Based on these metrics, it is then possible to define whether the models have behaved more or less correctly.

\subsection{Experiments}
We performed three different types of experiments.

The first one using as a transition model the dataset \texttt{big\_clean}, the associate perturbed dataset and test error 
models, and the language model \texttt{frequency-alpha-gcide}.

A second one using the same datasets but introducing a lemmatisation consisting in a simple dictionary lookup.

The last one using as transition model the dataset \texttt{lotr\_clean}, the associate perturbed dataset and test error 
models, and the language model \texttt{lotr\_language\_model}.\\

In all the experiments to come, reference will be made to the following variables:
\begin{itemize}
	\item p: the probability that a word has an edit
	\item ins: the probability that a word has a letter insertion
	\item del: the probability that a word has a letter deletion
	\item sub: the probability that a word has a letter substitution
	\item swap: the probability that a word has a swap between two letters
\end{itemize}

%%%%%%%%%%%%%%%%%%%%%%%%%%%%%%%%%
\subsubsection{Experiment 1}
The first experiment was carried out on two different hmm that differ in the choice of the \texttt{max\_edits} parameter.
In the first case, in fact, the edit distance considered was $1$, while in the second $2$.
In both cases the hmm is structured as follows:

\begin{figure}[H]
	\centering
	\begin{tabular}{ccccc}
		\toprule
				max states 	& language model	&  sentence ds  &  train typo ds 	&  test typo ds\\ \midrule
				\num{5} & \texttt{big\_language\_model} & \texttt{big\_clean}  & \texttt{big\_train}  &\texttt{big\_test}\\
		\bottomrule
	\end{tabular}
	\captionof{table}{Hmm model}
	\label{tab:error_model1}
\end{figure}

In the two tables to follow are shown the results obtained as regards the evaluation of the \textit{local correction} of our 
model on the typos dataset and the correction of the entire sequence with the \textit{Viterbi algorithm}.

\begin{figure}[H]
	\centering
	\begin{tabular}{lcc|cc}
		\toprule
		&\multicolumn{2}{c|}{\textsc{Edit Distance 1}} & \multicolumn{2}{c}{\textsc{Edit Distance 2}}\\
		& Train & Test & Train & Test \\
		\midrule
		\textbf{Num. observation} & \num{63759} & \num{15918} & \num{63759} & \num{15918} \\
		\textbf{Time (sec)}  & \num{53} & \num{14} & \num{1648} & \num{408} \\
		\textbf{Accuracy Top1} & \num{35,01}\%  & \num{35,60}\%  & \num{36,09}\%  & \num{36,84}\%  \\
		\textbf{Accuracy Top3} &  \num{46,24}\%  & \num{46,62}\%  & \num{48,61}\%  & \num{48,99}\%  \\
		\textbf{Accuracy Top5} & \num{50,19}\%  & \num{50,56}\%  & \num{52,86}\%  & \num{53,16}\%  \\
		\bottomrule
	\end{tabular}
	\captionof{table}{Typos performance evaluation}
	\label{tab:typo-eval1}
\end{figure}

\begin{figure}[H]
	\centering
	\begin{tabular}{lcccc}
		\toprule
		&\multicolumn{4}{c}{\textsc{Edit Distance 1}} \\
		\textbf{Dataset Perturbation} & \num{10}\%& \num{20}\% & \num{30}\%& \num{40}\%  \\
		\midrule
		\textbf{Time (sec)}							 &\num{201}			&\num{199}			& \num{191}			&\num{186} \\
		\textbf{Perturbed correct} 			   & \num{64,06}\% &\num{60,16}\%  & \num{57,30}\%	& \num{52,81}\% \\
		\textbf{Unperturbed not correct} &\num{41,70}\%	 &\num{43,36}\%  & \num{44,84}\% & \num{46,58}\% \\
		\textbf{Exact match} 					  &\num{2,60}\%	   &\num{2,32}\%	&\num{1,56}\%	&\num{1,42}\% \\
		\textbf{Accuracy} 							&\num{58,82}\%  &\num{57,47}\% &\num{55,73}\% &\num{42,60}\% \\
		\textbf{Precision}							 &\num{16,00}\% &\num{25,12}\% &\num{30,74}\%	&\num{33,75}\% \\
		\textbf{Recall}									&\num{79,66}\% &\num{77,36}\%&\num{74,30}\%	&\num{70,40}\%\\
		\textbf{F-Measure}						  &\num{46,28}\% &\num{48,49}\%&\num{49,65}\%	&\num{49,82}\%\\
		\bottomrule
		%\vspace*{0.5em}
	\end{tabular}
		\begin{center}
		...
		\end{center}
	\begin{tabular}{lcccc}
		\toprule
		&\multicolumn{4}{c}{\textsc{Edit Distance 2}} \\
		\textbf{Dataset Perturbation} & \num{10}\%& \num{20}\% & \num{30}\%& \num{40}\%  \\
		\midrule
		\textbf{Time (sec)}							 &\num{1110}		&\num{1121}	 	& \num{1116}		&\num{1121} \\
		\textbf{Perturbed correct} 			   & \num{54,96}\% &\num{55,47}\%  & \num{54,34}\%	& \num{51,76}\% \\
		\textbf{Unperturbed not correct} &\num{55,19}\%	 &\num{56,31}\%  & \num{57,52}\% & \num{58,43}\% \\
		\textbf{Exact match} 					  &\num{0,90}\%	   &\num{1,00}\%	&\num{0,76}\%	&\num{0,80}\% \\
		\textbf{Accuracy} 							&\num{45,81}\%  &\num{46,00}\% &\num{45,85}\% &\num{45,14}\% \\
		\textbf{Precision}							 &\num{10,67}\% &\num{18,45}\% &\num{23,7}\%	&\num{27,22}\% \\
		\textbf{Recall}									&\num{89,91}\% &\num{90,94}\%&\num{90,69}\%	&\num{88,21}\%\\
		\textbf{F-Measure}						  &\num{38,77}\% &\num{41,55}\%&\num{44,37}\%	&\num{46,18}\%\\
		\bottomrule
	\end{tabular}

	\captionof{table}{Sentences performance evaluation}
	\label{tab:sentence-eval1}
\end{figure}

%%%%%%%%%%%%%%%%%%%%%%%%%%%%%%%%%
\subsubsection{Experiment 2}
\begin{figure}[H]
	\centering
	\begin{tabular}{ccccc}
		\toprule
		p 				 & ins 				 	& del  				&  sub 			   &   swap\\ \midrule
		\num{0.5} & \num{0,70} & \num{0,70}  & \num{0,70} & \num{0,70}\\
		\bottomrule
	\end{tabular}
	\captionof{table}{Error Model}
	\label{tab:error_model2}
\end{figure}

We detected some problems with our dataset, in particular it lacks of plural forms and other things

In order to avoid this problem, we decided to try a new approach that use lemmatisation (stemming) ......

\begin{figure}[H]
	\centering
	\begin{tabular}{lcccc}
		\toprule
		& Time (sec)  & Accuracy Top1 & Accuracy Top3  &  Accuracy Top5 \\
		\midrule
		Train & \num{20986} & \num{41,38}\%  & \num{57,28} \% & \num{61,60} \% \\
		Test &	\num{5270}  & \num{41,54}\%  & \num{57,18} \% & \num{61,15} \%  \\
		\bottomrule
	\end{tabular}
	\captionof{table}{Typos performance evaluation}
	\label{tab:typo-eval2}
\end{figure}


\begin{figure}[H]
	\centering
	\begin{tabular}{ccccccc}
		\toprule
		\#sentence & Time (sec)  & Accuracy & Initial Error  &  Precision & Recall & Specificity \\
		\midrule
		\num{1000}	& \num{2705}  & \num{53,46}\%  & \num{15,01}\% & \num{90,96}\% & \num{54,50}\%  & 
		\num{10,52}\%  
		\\
		\bottomrule
	\end{tabular}
	\captionof{table}{Sentences performance evaluation}
	\label{tab:sentence-eval2}
\end{figure}

%%%%%%%%%%%%%%%%%%%%%%%%%%%%%%%%%
\subsubsection{Experiment 3}

As for the previous cases, the third experiment too was carried out on two different hmm with the same edit distances as 
the previous experiments. 
The parameters on which the structure is based are shown in the table below:

\begin{figure}[H]
	\centering
	\begin{tabular}{ccccc}
		\toprule
		max states 	& language model	&  sentence ds  &  train typo ds 	&  test typo ds\\ \midrule
		\num{5} & \texttt{lotr\_language\_model} & \texttt{lotr\_clean}  & \texttt{lotr\_train}  &\texttt{lotr\_test}\\
		\bottomrule
	\end{tabular}
	\captionof{table}{Hmm model}
	\label{tab:error_model3}
\end{figure}

In the two tables to follow are shown the results obtained as regards the evaluation of the \textit{local correction} of our 
model on the typos dataset and the correction of the entire sequence with the \textit{Viterbi algorithm}.

\begin{figure}[H]
	\centering
	\begin{tabular}{lcc|cc}
		\toprule
		&\multicolumn{2}{c|}{\textsc{Edit Distance 1}} & \multicolumn{2}{c}{\textsc{Edit Distance 2}}\\
		& Train & Test & Train & Test \\
		\midrule
		\textbf{Num. observation} & \num{49959} & \num{12570} & \num{49959} & \num{12570} \\
		\textbf{Time (sec)}  		& \num{32} 				& \num{9} 			& \num{1207} 	& \num{309} \\
		\textbf{Accuracy Top1} & \num{39,75}\%  & \num{40,47}\%  & \num{63,71}\%  & \num{63,87}\%  \\
		\textbf{Accuracy Top3} &  \num{44,74}\%  & \num{45,34}\%  & \num{76,44}\%  & \num{76,75}\%  \\
		\textbf{Accuracy Top5} & \num{46,25}\%  & \num{46,92}\%  & \num{80,86}\%  & \num{80,80}\%  \\
		\bottomrule
	\end{tabular}
	\captionof{table}{Typos performance evaluation}
	\label{tab:typo-eval3}
\end{figure}

\begin{figure}[H]
	\centering
	\begin{tabular}{lcccc}
		\toprule
		&\multicolumn{4}{c}{\textsc{Edit Distance 1}} \\
		\textbf{Dataset Perturbation} & \num{5}\%& \num{10}\% & \num{15}\%& \num{20}\%  \\
		\midrule
		\textbf{Time (sec)}							 &\num{116}			&\num{114}			& \num{114}			&\num{112} \\
		\textbf{Perturbed correct} 			   & \num{79,97}\% &\num{76,55}\%  & \num{73,74}\%	& \num{71,76}\% \\
		\textbf{Unperturbed not correct} & \num{13,68}\% &\num{14,46}\%  & \num{14,86}\%	& \num{15,69}\% \\
		\textbf{Exact match} 					  & \num{30,89}\% &\num{27,51}\%  & \num{25,23}\%	& \num{22,66}\% \\
		\textbf{Accuracy} 							&\num{85,99}\%  &\num{84,62}\% &\num{83,32}\% &\num{81,73}\% \\
		\textbf{Precision}							 & \num{28,99}\% &\num{41,97}\%  & \num{48,90}\%	& \num{52,85}\% \\
		\textbf{Recall}								  & \num{91,19}\% &\num{88,30}\%  & \num{86,09}\%	& \num{84,31}\% \\
		\textbf{F-Measure}						  & \num{74,19}\% &\num{72,87}\%  & \num{72,44}\%	& \num{71,97}\% \\
		\bottomrule
		%\vspace*{0.5em}
	\end{tabular}
	\begin{center}
		...
	\end{center}
	\begin{tabular}{lcccc}
		\toprule
		&\multicolumn{4}{c}{\textsc{Edit Distance 2}} \\
		\textbf{Dataset Perturbation} & \num{5}\%& \num{10}\% & \num{15}\%& \num{20}\%  \\
		\midrule
		\textbf{Time (sec)}							 &\num{866}			&\num{869}			& \num{854}			&\num{849} \\
		\textbf{Perturbed correct} 			   & \num{73,26}\% &\num{72,05}\%  & \num{71,10}\%	& \num{69,90}\% \\
		\textbf{Unperturbed not correct} & \num{18,47}\% &\num{18,86}\%  & \num{19,46}\%	& \num{20,16}\% \\
		\textbf{Exact match} 					  & \num{22,74}\% &\num{21,26}\%  & \num{19,52}\%	& \num{18,38}\% \\
		\textbf{Accuracy} 							&\num{81,07}\%  &\num{80,17}\% &\num{79,06}\% &\num{77,79}\% \\
		\textbf{Precision}							 & \num{22,40}\% &\num{34,49}\%  & \num{41,35}\%	& \num{45,37}\% \\
		\textbf{Recall}								  & \num{94,31}\% &\num{94,16}\%  & \num{93,63}\%	& \num{93,43}\% \\
		\textbf{F-Measure}						  & \num{68,76}\% &\num{68,60}\%  & \num{68,53}\%	& \num{68,58}\% \\
		\bottomrule
	\end{tabular}
	
	\captionof{table}{Sentences performance evaluation}
	\label{tab:sentence-eval3}
\end{figure}


%\textbf{FIXME}
%\begin{figure}[H]
%	\centering
%	\begin{tabular}{lcccc}
%		\toprule
%		& Time (sec)  & Accuracy Top1 & Accuracy Top3  &  Accuracy Top5 \\
%		\midrule
%		Train & \num{20986} & \num{41,38}\%  & \num{57,28} \% & \num{61,60} \% \\
%		Test &	\num{5270}  & \num{56,78}\%  & \num{74,75} \% & \num{80,59} \%  \\
%		\bottomrule
%	\end{tabular}
%	\captionof{table}{Typos performance evaluation}
%	\label{tab:typo-eval3}
%\end{figure}

% con 10% ho il 17% di errore
% con 10% ho il 25% di errore
% con 20% ho il 40% 31 di errore

%\begin{figure}[H]
%	\centering
%	\begin{tabular}{lccc|ccc}
%		\toprule
%		\textbf{Edit Distance} & \multicolumn{3}{c|}{1} & \multicolumn{3}{c}{2}\\
%		\textbf{Dataset Perturbation} & \num{10}\% & \num{15}\%& \num{20}\% & \num{10}\% & \num{15}\%& 
%		\num{20}\% \\
%		\midrule
%		Time (sec) &\num{165}&\num{157}& \num{147}&\num{2920}&\num{2835}&\num{2869}\\
%		\midrule
%		Perturbed correct & \num{81,33}\% &\num{76,48}\%& \num{71,58}\%& \num{61,33}\% 
%		&\num{60,99}\% 
%		&\num{59,91}\%\\
%		Unperturbed not correct &\num{41,62}\%&\num{42,34}\% & \num{43,94}\% & \num{61,75}\% & 
%		\num{61,61}\% & 
%		\num{62,09}\%\\
%		Exact match &\num{5,49}\%&\num{5,35}\%&\num{5,06}\%&\num{1,76}\%&\num{2,06}\%&\num{2,06}\%\\
%		Accuracy &\num{60,84}\% &\num{61,49}\% &\num{60,59}\% &\num{40,70}\% &\num{43,08}\% 
%		&\num{44,39}\% \\
%		Precision&\num{21,94}\%&\num{33,61}\% &\num{39,7}\%&\num{12,41}\%&\num{21,22}\%&\num{27,06}\%\\
%		Recall&\num{94,59}\%&\num{90,05}\%&\num{85,97}\%&\num{99,41}\%&\num{98,58}\%&\num{97,71}\%\\
%		F-Measure&\num{50,55}\%&\num{54,58}\%&\num{57,09}\%&\num{39,16}\%&\num{43,67}\%&\num{47,36}\%\\
%		\bottomrule
%	\end{tabular}
%	\captionof{table}{Sentences performance evaluation}
%	\label{tab:sentence-eval3a}
%\end{figure}
%
%
%\begin{figure}[H]
%	\centering
%	\begin{tabular}{ccccccc}
%		\toprule
%		\#sentence & Time (sec)  & Accuracy & Initial Error  &  Precision & Recall & Specificity \\
%		\midrule
%		\num{1620}	& \num{2705}  & \num{58,21}\%  & \num{17,39}\% & \num{90,30}\% & \num{58,62}\%  & 
%		\num{8,43}\%  
%		\\
%		\bottomrule
%	\end{tabular}
%	\captionof{table}{Sentences performance evaluation}
%	\label{tab:sentence-eval3}
%\end{figure}