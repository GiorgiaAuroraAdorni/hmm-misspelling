\chapter{Introduction}
\label{chap:Introduction}

In this project, we propose and evaluate a spelling correction algorithm. 
We modelled the typing process as a \textit{Hidden Markov Model} (HMM). 



%  keyboard
% and uses the probabilities that one character is typed for another, 



\section{Problem definition}
Given an observation sequence, typically a phrase,  and the model 
parameters, we are interested in detect and correct errors estimating the 
optimal state sequence. %FIXME

In our HMM, the hidden states represent the intended words and the 
observations are the typed words. 

The initial state probabilities $\pi$ are given by the word frequencies and 
the state transitions $A_{ij}$, that is the probability of one word given 
its predecessor obtained from ?? %FIXME.

The emission probabilities, $B_{ij}$, represent the probability of a typed 
word given the intended one, and depend on the confusion probabilities.

\subsection{Common Errors}
The following type of errors were considered:
\begin{itemize}
	\item \textsc{transposed adjacent characters}
	\item \textsc{omitted digit}
	\item \textsc{additional digit (twin errors)}:
	\item \textsc{substituted digit}: this is the probability to type a
	character $i$ when the character $j$ was intended ($P(i|j)$). This 
	probability was determined experimentally. %FIXME how? from where?
\end{itemize}

%We also detect the non-word error?

\section{Purpose}

\section{Design choice}
We considered that a typed word only depends on the previous one, being 
in the framework of Markov chains. If we know the probability of a word 
given its predecessor, the frequency of each word, and the probability to 
type word x when word y is intended, we have all the necessary ingredients 
to use Hidden Markov Models.

