\chapter{Introduction}
\label{chap:Introduction}

In recent decades, technology has had a strong impact on everyone's life.
It plays an important role in the communication process, simplifying 
different activities both individuals and businesses.

Nowadays we have advanced communication technology tools available, 
such as smartphones, tablets and computers that have simplified the way 
humans communicate. \\
Companies can write an e-mail and deliver it to all their consumers in a few 
minutes. \\
People can message their friends at every moment and share an interest 
with new friends from different countries.

This advancement in communication technology has made it necessary to 
equip our technological tools with a series of programs and software that 
control and correct automatically the misspelt words typed.

In this project, we propose and evaluate an automatic spelling correction 
algorithm, modelling the typing process as an \textit{Hidden Markov 
Model} (HMM). 

%  keyboard
% and uses the probabilities that one character is typed for another, 

\chapter{Problem formulation} % Representation
Given an observation sequence, typically a phrase, and the model 
parameters, we are interested in detect and correct errors estimating the 
optimal state sequence. %FIXME

In our HMM, the hidden states represent the intended words and the 
observations are the typed words. 

The initial state probabilities $\pi$ are given by the word frequencies and 
the state transitions $A_{ij}$, that is the probability of one word given 
its predecessor obtained from ?? %FIXME.

The emission probabilities, $B_{ij}$, represent the probability of a typed 
word given the intended one, and depend on the confusion probabilities.

%We also detect the non-word error?

\section{Purpose}

\section{Design choices}

We have chosen the English language for different reasons. First of all, the great majority of material in literature deals 
with the problem in question in the English languag. Moreover it is a simple language, both from a grammatical and a 
lexical point of view: it lacks in certain symbols, like accents and apostrophes, and genres. 
Furthermore all punctuation and special character symbols were not considered, but only letters and sometimes numbers. 
\\

We considered that a typed word only depends on the previous one, being 
in the framework of Markov chains. If we know the probability of a word 
given its predecessor, the frequency of each word, and the probability to 
type word x when word y is intended, we have all the necessary ingredients 
to use Hidden Markov Models.

\section{Softwares}
%FIXME
We have developed the project in \textbf{Python}.\\
The interface is a native macOS application written in \textbf{Swift}.
