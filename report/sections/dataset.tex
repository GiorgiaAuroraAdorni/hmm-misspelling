\chapter{Dataset}

%creazione del training set .. 

%Per ogni modello, la creazione dei set di training e validation è stata 
%realizzata utilizzando due tecniche diverse: in un primo esperimento è 
%stato 
%utilizzato un \textit{holdout} 80--20, in un secondo si è ricorso ad una 
%\textit{10-fold cross validation}.

\section{Dataset acquisition}
\label{section:dataset-exploration}
This project has resorted to the use of three different datasets. 

\subsection{Error Model}
This dataset was collected from the following resources 
\footnote{\url{https://www.dcs.bbk.ac.uk/~ROGER/corpora.html}}  
\footnote{\url{https://www.kaggle.com/rtatman/spelling-variation-on-urban-dictionary}}  
\footnote{\url{https://www.kaggle.com/bittlingmayer/spelling}}
\footnote{\url{http://luululu.com/tweet}}:
\begin{itemize}
	\item \textsc{birkbeck}: contains \num{36133} misspellings of \num{6136} words, taken from the native-speaker 
	section (British and American) of the Birkbeck spelling error corpus.
	\item \textsc{holbrook}: contains \num{1791} misspellings of \num{1200} words, taken from the book "English for the 
	Rejected" by 
	David Holbrook (Cambridge University Press - 1964).
	\item \textsc{aspell}: contains \num{531} misspellings of \num{450} words, taken from one assembled by Atkinson for 
	testing the 
	GNU Aspell spellchecker.
	\item \textsc{wikipedia}: contains \num{2455} misspellings of \num{1922} words, taken from the misspellings made by 
	Wikipedia 
	editors.
	\item \textsc{urban-dictionary-variants}: contains \num{716} variant spellings, taken from the text scraped from Urban 
	Dictionary (in UK English).
	\item \textsc{spell-set}: contains \num{670} typos.
	\item \textsc{tweet-typo}: contains \num{39172} typos, taken from Twitter.
\end{itemize}
\subsection{Language Model}
\subsection{Frequency Model}

\sections{analisi_esplorativa}
